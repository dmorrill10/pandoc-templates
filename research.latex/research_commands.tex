% Custom configuration
%=====================

\usepackage{enumerate}
\usepackage{multirow}

% macros
%-------

\newcommand{\pdLeq}{\preceq}
\newcommand{\naive}{na\"{\i}ve}
\def\forceLatex#1{#1}

\newcommand{\final}{\fbox}
\newcommand{\bs}{\boldsymbol}
\newcommand{\tr}{\operatorname{tr}}
\newcommand{\abs}[1]{\left\lvert#1\right\rvert}
\newcommand{\norm}[1]{\left\lVert#1\right\rVert}
\newcommand{\set}[1]{\left\{ {#1} \right\}}

\newcommand{\ip}[2]{\langle #1,#2 \rangle}
\newcommand{\diag}[1]{\mathop{\bigtriangleup\subex{#1}}\nolimits}
\newcommand{\stack}[1]{\operatorname{stack}\subex{#1}}
\newcommand{\deriv}[2]{\frac{d #1}{ d #2 }}
\newcommand{\pderiv}[2]{\frac{\partial #1}{ \partial #2 }}
\newcommand{\secondPderiv}[3]{\frac{\partial^2 #1}{ \partial #2 \partial #3 }}
\newcommand{\grad}{\nabla}

\newcommand{\subex}[1]{\left(#1\right)}
\newcommand{\subblock}[1]{\left[#1\right]}
\newcommand{\inTextMatrix}[1]{\subblock{\begin{smallmatrix}#1\end{smallmatrix}}}

\usepackage{amsthm}
\newtheorem{theorem}{Theorem}[section]
\newtheorem{definition}[theorem]{Definition}
\newtheorem{corollary}[theorem]{Corollary}
\newtheorem{lemma}[theorem]{Lemma}

\newcommand{\avg}[1]{\bar{#1}}
\newcommand{\smallText}[1]{\tiny {#1} \normalsize}
\newcommand{\estimate}[1]{\tilde{#1}}
\newcommand{\alternate}[1]{\hat{#1}}

\newcommand{\KL}[2]{\operatorname{KL}\subex{#1 \;\|\; #2}}
\newcommand{\bigO}[1]{\operatorname{\mathcal{O}}\subex{#1}}
\newcommand{\smallo}[1]{\operatorname{o}\subex{#1}}

\newcommand{\ones}{\bs{1}}

% Thanks to Csaba Szepesvari for these
\usepackage{algorithm}
\usepackage{algpseudocode}
\newcommand{\bookbox}[1]{
    \par\noindent
    \begin{center}
    \framebox[.94\textwidth]{
    \begin{minipage}{0.82 \textwidth}{#1}
    \end{minipage} }
    \end{center}
    \par\noindent }

\newcommand{\bookboxt}[2]{
    \bookbox{
      \begin{center}
        \sc #1
      \end{center}
     \medskip
      #2 }
}
\newcommand{\algDesc}{\bookboxt}

\newcommand{\E}[1]{\operatorname{\mathbb{E}}\subblock{#1}}
\newcommand{\Ewrt}{\operatorname{\mathbb{E}}}
\newcommand{\I}[1]{\operatorname{\mathbb{I}}_{#1}}
\newcommand{\Prob}[1]{\operatorname{\mathbb{P}}\subex{#1}}
\newcommand{\LogicalOr}{\vee}
\newcommand{\LogicalAnd}{\wedge}
\newcommand{\newln}{\\&\quad\quad\quad{}}
\newcommand{\alg}[1]{\textsc{#1}}
\newcommand{\problem}[1]{\textsc{#1}}
\newcommand{\reals}{\mathbb{R}}
\newcommand{\integers}{\mathbb{Z}}
\newcommand{\gaussian}[2]{\mathcal{N}\subex{#1,#2}}

\usepackage{braket}

\usepackage{ mathrsfs }
\newcommand{\lagrangian}{\mathscr{L}}
\newcommand{\regularizer}{\mathcal{R}}
\newcommand{\nnOrth}{\reals{}_+}
\newcommand{\sumOverRoundsToTMinusOne}{\sum_{s=1}^{t-1}}
\newcommand{\sumOverRoundsToTMinusTwo}{\sum_{s=1}^{t-2}}
\newcommand{\sumOverRounds}{\sum_{t=1}^n}
\newcommand{\sumOverExperts}{\sum_{i=1}^d}
\newcommand{\sumOverExpertsMinusOne}{\sum_{i=1}^{d-1}}
\newcommand{\sumOverExpertsMinusOneJ}{\sum_{j=1}^{d-1}}
\newcommand{\sumOverExpertsMinusOneK}{\sum_{k=1}^{d-1}}
\newcommand{\sumOverExpertsJ}{\sum_{j=1}^d}
\newcommand{\experts}{\mathcal{E}}
\newcommand{\paths}{\mathcal{P}}
\newcommand{\vertices}{V}
\newcommand{\edges}{E}
\newcommand{\edge}{x}
\newcommand{\numPathsUpperBound}{2^{\abs{\vertices} - 2}}
\newcommand{\decisionSet}{D}
\newcommand{\lossSet}{\mathcal{L}}
\newcommand{\loss}{\ell}
\newcommand{\utilitySet}{\mathcal{U}}
\newcommand{\utility}{u}
\newcommand{\forecasterCumulativeLoss}{\hat{L}}
\newcommand{\lossCodomain}{\subblock{0,1}}
\newcommand{\expertWeight}{w}
\newcommand{\normalizedExpertWeight}{\overline{w}}
\newcommand{\updateFactors}{b}

\newcommand{\altLoss}{\estimate{\loss}}
\newcommand{\prefix}{\;\sqsubseteq\;}
\newcommand{\strictPrefix}{\;\sqsubset\;}

\allowdisplaybreaks

\usepackage[obeyFinal]{todonotes}
\usepackage{booktabs}
\usepackage{listings}

% Thanks to Ed Woodcock (http://stackoverflow.com/questions/586572/make-code-in-latex-look-nice) for this.
\usepackage{color}

\definecolor{mygreen}{rgb}{0,0.6,0}
\definecolor{mygray}{rgb}{0.5,0.5,0.5}
\definecolor{mymauve}{rgb}{0.58,0,0.82}

\usepackage{setspace}

\lstset{ %
  belowcaptionskip=1\baselineskip,
  backgroundcolor=\color{white},   % choose the background color; you must add \usepackage{color} or \usepackage{xcolor}
  basicstyle={\footnotesize\singlespacing},aboveskip={-4ex},        % the size of the fonts that are used for the code
  breakatwhitespace=false,         % sets if automatic breaks should only happen at whitespace
  breaklines=true,                 % sets automatic line breaking
  % captionpos=b,                    % sets the caption-position to bottom
  commentstyle=\color{mygreen},    % comment style
  deletekeywords={...},            % if you want to delete keywords from the given language
  escapeinside={\%*}{*)},          % if you want to add LaTeX within your code
  extendedchars=true,              % lets you use non-ASCII characters; for 8-bits encodings only, does not work with UTF-8
  frame=single,                    % adds a frame around the code
  keepspaces=true,                 % keeps spaces in text, useful for keeping indentation of code (possibly needs columns=flexible)
  keywordstyle=\color{blue},       % keyword style
  language=Python,                 % the language of the code
  morekeywords={*,...},            % if you want to add more keywords to the set
  numbers=none,                    % where to put the line-numbers; possible values are (none, left, right)
  numbersep=5pt,                   % how far the line-numbers are from the code
  numberstyle=\tiny\color{mygray}, % the style that is used for the line-numbers
  rulecolor=\color{black},         % if not set, the frame-color may be changed on line-breaks within not-black text (e.g. comments (green here))
  showspaces=false,                % show spaces everywhere adding particular underscores; it overrides 'showstringspaces'
  showstringspaces=false,          % underline spaces within strings only
  showtabs=false,                  % show tabs within strings adding particular underscores
  stepnumber=2,                    % the step between two line-numbers. If it's 1, each line will be numbered
  stringstyle=\color{mymauve},     % string literal style
  tabsize=2,                       % sets default tabsize to 2 spaces
  title=\lstname                   % show the filename of files included with \lstinputlisting; also try caption instead of title
}

\linespread{1.2}

\expandafter\def\expandafter\normalsize\expandafter{%
    \normalsize
    \setlength\abovedisplayskip{0pt}
    \setlength\belowdisplayskip{15pt}
    % \setlength\abovedisplayshortskip{0pt}
    % \setlength\belowdisplayshortskip{0pt}
}

\usepackage{times}
\usepackage{helvet}
\usepackage{courier}
\usepackage{longtable}
\usepackage{caption}

\setcounter{secnumdepth}{0}

\usepackage{amsmath,amsthm,amssymb,mathtools}
\usepackage{algorithm,algorithmicx,algpseudocode}

\newcommand{\defword}[1]{\textbf{\boldmath{#1}}}
\newcommand{\argmin}{\operatornamewithlimits{argmin}}
\newcommand{\argmax}{\operatornamewithlimits{argmax}}
\DeclareRobustCommand{\rmPlus}{%
    \ifmmode
        \text{RM}^+
    \else
        \GenericError{\space\space\space\space}
            {Attempt to use \@backslashchar rmPlus outside of math mode}
            {See my preamble documentation for explanation.}
            {Need to use either use inline or display math.}%
    \fi
}
\DeclareRobustCommand{\rcfrPlus}{%
    \ifmmode
        \text{RCFR}^+
    \else
        \GenericError{\space\space\space\space}
            {Attempt to use \@backslashchar rcfrPlus outside of math mode}
            {See my preamble documentation for explanation.}
            {Need to use either use inline or display math.}%
    \fi
}
\DeclareRobustCommand{\rrmPlus}{%
    \ifmmode
        \text{RRM}^+
    \else
        \GenericError{\space\space\space\space}
            {Attempt to use \@backslashchar rrmPlus outside of math mode}
            {See my preamble documentation for explanation.}
            {Need to use either use inline or display math.}%
    \fi
}
\newcommand{\featureExp}{\varphi}
\newcommand{\features}[1]{\featureExp \subex{#1}}

\usepackage{ upgreek }
\newcommand{\featureSpace}{\Upphi}

\newcommand{\inputSpace}{\mathcal{X}}
\newcommand{\histories}{\mathcal{H}}
\newcommand{\infoSets}{\mathcal{I}}
\newcommand{\infoSet}{I}
\newcommand{\terminalHistories}{\mathcal{Z}}
\newcommand{\terminalHistory}{z}
\newcommand{\history}{h}
\newcommand{\Iabs}{I^{\text{abstract}}}
\newcommand{\Ifull}{I^{\text{full}}}
\newcommand{\actions}{A}
\newcommand{\action}{a}
\newcommand{\numActions}{N}
\newcommand{\zeros}{\bs{0}}
\newcommand{\reward}{v}
\newcommand{\policy}{\sigma}
\newcommand{\policySpace}{\Sigma}
\newcommand{\regret}{r}
\newcommand{\Regret}{R}
\newcommand{\qRegret}{q}
\newcommand{\QRegret}{Q}
\newcommand{\maxNumRounds}{T}
\newcommand{\currentRound}{t}
\newcommand{\playerChoice}{p}
\newcommand{\chance}{c}
\newcommand{\util}{u}
\newcommand{\reachProb}{\pi}
\newcommand{\gap}{\varepsilon}
\newcommand{\altGap}{\epsilon}
\newcommand{\deviation}{\delta}
\newcommand{\arbVector}{w}
\newcommand{\maxReward}{L}
\newcommand{\rewardSet}{\mathcal{V}}
\newcommand{\learner}{\mathcal{A}}
\newcommand{\linkFn}{\phi}
\newcommand{\LinkFn}{\Phi}
\newcommand{\emptySet}{\emptyset}
\newcommand\floor[1]{\lfloor#1\rfloor}
\newcommand\ceil[1]{\lceil#1\rceil}
\newcommand\intersect{\cap}


\newcommand{\N}[1]{\text{\textbf{\boldmath{#1}}}}
\newcommand{\F}[1]{\operatorname{#1}}
\newcommand{\DeltaMin}{\Delta \text{Min}}
\newcommand{\PhiSum}{\Phi \text{Sum}}

\usepackage{float}

\usepackage{subcaption}
\captionsetup{compatibility=false}
